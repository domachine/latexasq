% -*- TeX-master: "main.tex" -*-

\section{Einführung}

Dieses Dokument beschäftigt sich mit dem sogenannten "`Double Bridge"'
Experiment. Hierbei handelt es sich um eine Möglichkeit, das Phenomen
zu erforschen, warum beispielsweise Ameisen in größeren Gruppen in der
Lage sind, den kürzesten Weg zu einer Futterstelle auszumachen.

Dieses Phänomen ist durchaus interessant für die Informatik, da das
Verhalten der Ameisen für intelligente Algorithmen verwendet werden
kann. Sehr naheliegend scheinen da natürlich Algorithmen zur
Koordination von mehreren Robotern. Allerdings scheint der Algorithmus
der Ameisen geradezu genial zu sein um den kürzesten Weg zu etwas in einem
unbekannten Gebiet zu finden.