% -*- TeX-master: "main.tex" -*-

\section{Implementierungsdetails}

Der Versuch wird durch ein Javascript Programm nachgestellt, das einen
Graphen in Baumform aufbaut und zufällig „Ameisen” erzeugt, um
diese auf dem Startknoten zu platzieren. Dann wird der Zustand des Graphen
immer schrittweise geändert. Dabei wird für jede einzelne Ameise auf
jedem Knoten per Zufall entschieden, welchen Weg diese
einschlägt. Dieser Zufall wird beeinflusst von den auf den Knoten
anliegenden „Pheromonspuren”. Diese werden erzeugt, wenn eine Ameise
einen Knoten überkreuzt hat. Um den Zustand zu ändern, wird eine
Funktion \emph{step} verwendet, die in einem bestimmten Takt
aufgerufen wird (Siehe \ref{lstStepfn}).

Die Funktion steigt rekursiv in den kompletten Pfad ab und ändert wie
oben beschrieben selbigen ab.

\begin{lstlisting}[captionpos=b, caption=Step Funktion, label=lstStepfn]
exports.step = function (root) {
    var stepcount = root.stepcount++;

    var _process = function (node) {
        for (var i in node.ants) {
            //just forward gooing ants
            var ant = node.ants[i];

            if(ant.stepcount > stepcount)
                continue;

            var next = nextNode(ant, node);
            if(next !== undefined){
                //set ant on next node
                next.ants.push(ant);
                node.ants[i] = undefined;

                ant.returnPath.push(node);
                next.pheromone++;

                ant.stepcount++;
            } else {
                node.antsOnHomerun.push(ant);
                node.ants[i] = undefined;
            }
        }

        for (var i in node.antsOnHomerun){
          // Zuruecklaufende Ameisen werden behandelt
          // Aehnlich wie suchende Ameisen behandelt
          .
          .
          .
        }

        .
        .
        .

        for (var i in node.edges)
            //recursive process subnodes
            if(node.edges[i] !== undefined)
                _process(node.edges[i]);
    };

    _process (root);
}
\end{lstlisting}

Um das ganze nun anschaulisch und für Expermente zugänglich wird das
Werkzeug \emph{Graphviz} verwendet um den Pfad nach jedem Schritt zu
visualisieren. Hierfür werden sogenannte \emph{DOT} Dateien
geschrieben, die Graphviz einlesen und interpretieren
kann. Verwendet werden hierfür die Funktionen \emph{draw} (siehe
\ref{lstDrawfn}) und \emph{save} (siehe \ref{lstSave}).

  \begin{lstlisting}[captionpos=b, caption=Draw Funktion, label=lstDrawfn]
exports.draw = function(node, fathernodetext, depth){
    var nodetext = "\"Node " +
               node.id +
               "\\n(Ants: " +
               node.ants.length +
               "/" +
               node.antsOnHomerun.length + ")\"";
    var output = "";
    for(var i in node.edges){
        output += exports.draw(node.edges[i], nodetext, depth+1);
    }
    var color = "";
    if(node.edges.length == 0){
        color = nodetext + " [style=filled,color=\"gray\"];\n\t";
    }
    if(fathernodetext !== undefined)
        return color + fathernodetext +
               " -> " + nodetext +
               " [label=" + node.pheromone + "];\n\t " +
               output;
    else
        return "\n\t "+output;
}
  \end{lstlisting}

  \begin{lstlisting}[captionpos=b, caption=Mehr Text, label=lstSave]
exports.save = function(root, id){

    var output = "digraph G {\n";
    output += "\t" + exports.draw(root, undefined, 0);
    output += "\n}";

    var exec = require('child_process').exec;
    function puts(error, stdout, stderr) { }

    var filename = "graph_" + id + "_" + root.stepcount;
    exec("echo '"+output+"' > " + filename + ".dot", puts);
    exec("dot -Tpng " + filename + ".dot -o " + filename + ".png", puts);
}
  \end{lstlisting}

