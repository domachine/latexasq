% -*- TeX-master: "main.tex" -*-

\section{Versuchsbeschreibung}

Der Versuchsaufbau sieht wie folgt aus: Es existiert ein Graph mit
zufällig mit einander verketteten Knoten. Dieser Graph wird im
folgenden als „Pfad” bezeichnet. Auf diesem Pfad werden Ameisen
platziert. Wenn eine Ameise nun auf einen Knoten kommt, hinterlässt
sie auf diesem eine Pheromonspur. Diese Pheromonspur beeinflusst
nachfolgende Ameisen in ihrer Wegwahl. Jede Ameise entscheidet zwar
bei jeder Weggabelung zufällig welchen Weg sie einschlägt. Die
Wahrscheinlichkeit für einen Weg steigt jedoch erheblich, wenn der
nächste Knoten eine starke Pheromonspur enthält. Sprich, wenn dieser
schon häufiger durch die gleiche oder andere Ameisen besucht wurde.
Die Idee ist hierbei, dass die Ameisen auf diese Weise in der Lage
sind, den kürzesten Weg zu einer Futterquelle zu finden. Wenn eine
Ameise nämliche die Futterquelle gefunden hat, wird sie umkehren und
den selben Weg wieder zurück laufen. Und somit die Pheromonspur auf
diesem Weg verstärken. Wenn das nun mehrere Ameisen hintereinander
tun, wird sich die Pheromonspur so weit verstärken, dass die
Wahrscheinlichkeit, dass dieser Weg eingeschlagen wird gegen eins
streben wird.
